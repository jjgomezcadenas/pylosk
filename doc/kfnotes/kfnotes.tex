\documentclass[10pt]{article}
%\pagestyle{headings}

\oddsidemargin 0.0in
\evensidemargin 0.0in
\textwidth 6.0in
\topmargin 0.2in
\footskip 20pt
\textheight 8.5in

\usepackage{hyperref,color,graphicx,braket,mathrsfs,slashed,amsmath,amssymb}
\hypersetup{colorlinks,%
            citecolor=blue,%
            linkcolor=blue,%
            filecolor=blue,%
            urlcolor=red}

\bibliographystyle{plain}

\begin{document}

{\noindent\Large\textbf{Notes on the Kalman Filter Formalism and Implementation}}\\
J.J., Francesc, Josh\\
\today

\section{The Kalman Filter}
\noindent The Kalman Filter is a technique that can be used to iteratively fit a dataset assuming some predictive model and a noise process that distorts
the evolution of the system from that dictated by the model.  It provides the best fit, or
that yielding the minimal mean square error in the fit parameters, given that the noise processes involved are Gaussian in nature.  The fit parameters
are arranged in a vector called the state vector, and the uncertainties and correlations between them are encompassed in a covariance matrix.  The
state vector and covariance matrix are determined recursively using the measurements in the dataset, beginning with some initial values, and for each 
recorded measurement predicting a new state vector and correlation matrix using a model and then correcting (or ``filtering'') these predictions based 
on the measured value.  Here we will be most interested in the application of the Kalman Filter to the fitting of ionization tracks left by energetic
electrons in high pressure xenon gas.\\

\noindent Here we state the standard Kalman Filter equations to be applied to a dynamical system whose evolution is governed by a known predictive model
with an accompanied noise process.  We adopt a notation similar to that of \cite{Wolin_1993}.  We denote the state vector 
containing the values of the fit parameters as $\mathbf{a}$, the covariance matrix containing the uncertainties and correlations in these parameters
as $\mathbf{C}$, the vector containing the results of a single measurement in the dataset as $\mathbf{m}$, and the covariance matrix of the measured values
as $\mathbf{G}$.  In addition, the noise process distorts the values of the fit parameters according to a covariance matrix $\mathbf{Q}$.  The Kalman Filter
operates in two stages on each measurement $k$ in a dataset of $N$ measurements.  In the first stage, the fit parameters of the previous state are 
propagated to new values, forming a predicted state.  In the second stage, this state is modified based on the value of the measurement to give
a ``filtered'' state.\\

\noindent For a dataset of $N$ measurements we will have $N-1$ predicted states and $N-1$ filtered states, and we label the states
with the index $k$.  We denote the predicted state and covariance matrix as $\mathbf{a}_{\mathrm{P},k}$ and $\mathbf{C}_{\mathrm{P},k}$, and the 
filtered state and covariance matrix as $\mathbf{a}_{\mathrm{F},k}$ and $\mathbf{C}_{\mathrm{F},k}$.  The predicted state vector and covariance matrix
are calculated as

\begin{equation}
 \begin{split}
  \mathbf{a}_{\mathrm{P},k} = & \,\,\mathbf{F}_{k-1}\mathbf{a}_{\mathrm{F},k-1},\\
  \mathbf{C}_{\mathrm{P},k} = & \,\,\mathbf{F}_{k-1}\mathbf{C}_{\mathrm{F},k-1}\mathbf{F}_{k-1}^{T} + \mathbf{Q}_{k-1},
 \end{split}
\end{equation}

% Note that here $\mathbf{Q}_{k-1}$ means the covariance matrix constructed with the coordinates of the filtered state (k-1) and z = zi of the current
% slice k.  In addition, the dz of the current slice k is used to compute the multiple scattering for this slice.  In other words, we are sitting on the
% scattering surface for state k (the beginning of the slice) when we compute $\mathbf{Q}_{k-1}$
\noindent where $\mathbf{F}_{k-1}$ is the propagator matrix which advances the state vector from state $k-1$ to the predicted state $k$ according to the
predictive model.  Note that in the $k = 0$ case, an initial state is determined by some arbitrary means and is set as the filtered state to begin the process.  
For a measurement $\mathbf{m}_{k}$ with measurement covariance matrix $\mathbf{V}_{k}$ and inverse $\mathbf{G}_{k} = \mathbf{V}_{k}^{-1}$, the filtered 
state vector and covariance matrix are calculated as

\begin{equation}
 \begin{split}
  \mathbf{a}_{\mathrm{F},k} = & \,\,\mathbf{a}_{\mathrm{P},k} + \mathbf{K}_{k}(\mathbf{m}_{k} - \mathbf{H}_{k}\mathbf{a}_{\mathrm{P},k}),\\
  \mathbf{C}_{\mathrm{F},k} = & \,\,[\mathbf{C}_{\mathrm{P},k}^{-1} + \mathbf{H}_{k}^{T}\mathbf{G}_{k}\mathbf{H}_{k}]^{-1},
 \end{split}
\end{equation}

\noindent where $\mathbf{I}$ is the identity matrix, $\mathbf{H}_{k}$ converts the state vector $\mathbf{a}_{\mathrm{P},k}$ into a corresponding physical measurement, 
and $\mathbf{K}_{k}$ is called the Kalman gain matrix,

\begin{equation}
 \mathbf{K}_{k} = \bigl[\mathbf{C}_{\mathrm{P},k}^{-1} + \mathbf{H}_{k}^{T}\mathbf{G}_{k}\mathbf{H}_{k}\bigr]^{-1}\mathbf{H}_{k}^{T}\mathbf{G}_{k}.
\end{equation}

\section{3-D Track Reconstruction with the Kalman Filter}
\noindent Here we detail the formalism for reconstructing electron tracks in high pressure xenon from a dataset consisting of a list of $(x,y,z)$ hits.
We represent the track determined by $N$ measurements by a series of $N-1$ nodes, each slice $k$ containing the $(x,y)$ coordinates of measurement $k$
and an initial z-coordinate $z_{i}$ from measurement $k$ and final z-coordinate $z_{f}$ from measurement $k+1$.  We write a state vector that, at each node,
contains the information about the $(x,y)$ position of the track and the direction of the particle at that point in the track.  Note that all state vectors
are associated with a node, and therefore we assume that the state before propagation across the node has $z = z_{i}$ and after propagation has
$z = z_{f}$.  This requires 4 degrees of freedom, including x, y, and two angles we will take to be the tangent angles with the $\mathbf{\hat{z}}$-axis,
$\tan\theta_{x} = dx/dz$ and $\tan\theta_{y} = dy/dz$.  Rather than using the coordinates $x$ and $y$ themselves as two of the state vector components,
we will choose the following state representation from \cite{Wolin_1993}

\begin{equation}\label{eqn_avector}
 \mathbf{a} = \left(\begin{array}{c} P_{1}\\ P_{2}\\ P_{3}\\ P_{4} \end{array} \right) = \left(\begin{array}{c} x - z\tan\theta_{x}\\ y - z\tan\theta_{y}\\ \tan\theta_{x}\\ \tan\theta_{y} \end{array} \right),
\end{equation}

\noindent where $z = z_{f}$ of the associated slice.  For a given node $k$, the predictive model will propagate the filtered state at index 
$k-1$ (with $z = z_{i}$ of the node $k$, equivalent to $z_{f}$ of node $k-1$) to the predicted state at index $k$, assuming the state travels
in a straight line across a distance $dz$, so that $dx = (dx/dz)dz = (\tan\theta_{x})dz$, $P_{1}$ is propagated as

\begin{equation}
 P_{1} = x - z\tan\theta_{x} \rightarrow (x + dz\tan\theta_{x}) - (z + dz)\tan\theta_{x} = x - z\tan\theta_{x}.
\end{equation}

\noindent That is, in propagating from $z_{i}$ to $z_{f}$ the parameter $P_{1}$ remains unchanged in this representation.  Similar equations can be written 
for $P_{2}$.  Since the particle is assumed to move in a straight line across $dz$, the direction tangents $P_{3}$ and $P_{4}$ are also unchanged.  We conclude
therefore that our propagation matrix for all values of $k$ is simply the identity,

\begin{equation}
 \mathbf{F}_{k} = \mathbf{I}.
\end{equation}

\noindent The use of the chosen representation (equation \ref{eqn_avector}) has allowed for this simplification, but now we must write the covariance matrix 
of the multiple scattering noise process $\mathbf{Q}_{k}$ in this representation.  Multiple scattering is assumed to occur during the travel from $z_{i}$
to $z_{f}$ in which the angles $\theta_{x}$ and $\theta_{y}$ are altered by gaussian noise with variance \cite{RPP_2012}

\begin{equation}
 \sigma^{2}(\theta_{x,y}) = \frac{13.6\,\,\mathrm{MeV}}{\beta p}\sqrt{dz/L_{0}}\bigl[1 + 0.038\ln(dz/L_{0})\bigr],
\end{equation}

\noindent for an electron with momentum $p$ in MeV/c at the current node, beta factor $\beta$ (where $c = 1$).  $L_{0}$ is the radiation length of the
detector medium (for xenon we use $L_{0} = 1530 \,\mathrm{cm} / P$, where $P$ is the pressure in bar).\footnote{This formula is stated to give results of 
11\% accuracy or better for $10^{-3} < dz/L_{0} < 100$, so it should be ensured that the nodes have widths that fit in this range.}
The two independent variances on $\theta_{x}$ and $\theta_{y}$ can be propagated to variance in the state parameters, which all depend on at least
one of the two angles.  This is done in \cite{Wolin_1993} and the resulting correlation matrix is

\begin{equation}
\begin{split}
 \mathbf{Q} = \sigma^{2} & (\theta_{x,y})[1 + (P_{3})^{2} + (P_{4})^{2}] \\\\
  & \times \left(\begin{array}{cccc} z^{2}[1 + (P_{3})^{2}] & z^{2}P_{3}P_{4} & -z[1 + (P_{3})^{2}] & -zP_{3}P_{4}\\ z^{2}P_{3}P_{4} & z^{2}[1 + (P_{4})^{2}] & -zP_{3}P_{4} & -z[1 + (P_{4})^{2}]\\ -z[1 + (P_{3})^{2}] & -zP_{3}P_{4} & [1 + (P_{3})^{2}] & P_{3}P_{4}\\ -zP_{3}P_{4} & -z[1 + (P_{4})^{2}] & P_{3}P_{4} & [1 + (P_{4})^{2}] \end{array} \right),
\end{split}
\end{equation}

\noindent noting that $z$ in the above expression is the $z$-coordinate at which the scatter occurs ($z_{i}$ of the node).\\

\noindent We can obtain the $\mathbf{H}$ matrix by asking how we would convert a state $\mathbf{a}$ into an $(x,y)$ measurement at that node.  This
amounts to extracting the values $x$ and $y$, and

\begin{equation}
\begin{split}
 x = P_{1} + zP_{3}\\
 y = P_{2} + zP_{4},
\end{split}
\end{equation}

\noindent where $z = z_{f}$ for the node since we will be acting with $\mathbf{H}$ on the filtered state.  Therefore we have

\begin{equation}
 \mathbf{H} = \left(\begin{array}{cccc} 1 & 0 & z & 0\\ 0 & 1 & 0 & z\end{array} \right),
\end{equation}

\noindent where again $z = z_{f}$ for the given node (so $\mathbf{H}$ will depend on $k$).  Assuming the x and y measurements are uncorrelated with
variances $\sigma_{x}^{2}$ and $\sigma_{y}^{2}$, the measurement covariance matrix will be diagonal, and therefore its inverse will also be diagonal,

\begin{equation}
 \mathbf{G} = \left(\begin{array}{cc} 1/\sigma_{x}^{2} & 0\\ 0 & 1/\sigma_{y}^{2}\end{array}\right).
\end{equation}

\noindent The above equations can be used in the Kalman Filter formalism for the application of a straight-line fit to a track with multiple scattering noise.

\bibliography{kfnotes}
\end{document}
